% section/paper on sampling with sampled priors.
% \documentclass[preprint,linenumbers]{aastex631} # 6.31 fails with \align!!
\documentclass[linenumbers, onecolumn]{aastex63}
\usepackage{xcolor}

\usepackage{natbib}
\usepackage{amsmath}
\usepackage{enumitem}

\shortauthors{Bernstein et al.}

\newcommand{\ie}{\textit{i.e.}}
\newcommand{\eg}{\textit{e.g.}}
\newcommand{\E}{\mathrm{E}}
\newcommand{\eqq}[1]{Equation~(\ref{#1})}
\newcommand\gary[1]{{\color{red} \{\textbf{GMB}: #1\}}}
%\tabletypesize{\footnotesize}
\newcommand{\vecc}{\ensuremath{\mathbf{c}}}
\newcommand{\vecq}{\ensuremath{\mathbf{q}}}
\newcommand{\vecn}{\ensuremath{\mathbf{n}}}
\newcommand{\vecu}{\ensuremath{\mathbf{u}}}
\newcommand{\hatc}{\ensuremath{\hat{\mathbf{c}}}}
\newcommand{\vecP}{\ensuremath{\mathbf{P}}}
\newcommand{\covm}{C}
\newcommand{\matA}{\ensuremath{A}}
\newcommand{\matD}{D}
\newcommand{\matU}{U}
\newcommand{\matV}{V}
\newcommand{\matW}{W}
\newcommand{\proj}{P}  % Projection matrix
\newcommand{\jac}{J}   % Jacobian
\newcommand{\ident}{I}  % identity
\newcommand{\DD}{\Delta_D}
\newcommand{\likeli}{\mathcal{L}}
\newcommand{\trace}{\text{Tr}}
\begin{document}

\title{Sampling Bayesian probabilities given only sampled priors}

\author[0000-0002-8613-8259]{Gary M. Bernstein}
\affiliation{Department of Physics and Astronomy, University of Pennsylvania, Philadelphia, PA 19104, USA}
\email{garyb@upenn.edu}

\author{Troxel,\ldots}

\begin{abstract}
	\vspace{0.2in}
Later.
\end{abstract}
\reportnum{}

\section{Motivation} \label{sec:intro}

Our motivating situation is that we have a vector of observable summary statistics \vecc, such as the binned 2-point correlation functions of cosmic fields, that we are using to constrain a set of parameters of interest \vecq, which are cosmological parameters such as $\Omega_m, \sigma_8,$ etc.  There is a model $\vecc(\vecq,\vecn)$ for the observables which involves the parameters of interest, but also a vector \vecn\ of nuisance parameters, which include the coefficients of some linear expansion of the redshift distributions $n(z)$ of some population of the galaxies being observed:
\begin{equation}
  n(z) = \sum_{k=1}^{M} n_k b_k(z).
  \label{eq:nzbasis}
\end{equation}
The $b_k$ are a set of predetermined basis functions for the redshift distribution.  In our case there are multiple galaxy populations to be characterized, leading to hundreds of parameters $n_k$ to be considered.

We wish to characterized the Bayesian posterior probability
\begin{equation}
  p(\vecq | \vecc) \propto \int dn\, \likeli(\vecc | \vecq, \vecn) p(\vecq) p(\vecn),
\label{eq:posterior}
\end{equation}
where $\likeli(\vecc | \vecq, \vecn)$ is a known likelihood function of the data, and $p(\vecq)$ and $p(\vecn)$ are priors on the parameters.  This posterior is complex enough that it requires approximation by the output of a Markov chain wandering across the space $(\vecq,\vecn).$

The scenario of interest is when \emph{the prior $p(\vecn)$ is itself known only from a set of samples of $\vecn$ from this distribution.} Most MC samplers require that the posterior be an evaluable function of any value of the parameters, and it is the general task of density estimators to convert the samples of $\vecn$ into an evaluable $p(\vecn).$  But when \vecn\ is of high dimension, two problems arise: first, there may be insufficient available samples to create a viable density estimator; second, sampling of the posterior in (\ref{eq:posterior}) becomes infeasible if the MC must traverse a high-dimensional space.

One approach would be to run a new MC chain over \vecq\ for each of the samples we have of \vecn, and then concatenate these to effect marginalization over \vecn.  This is clearly infeasible if a large number of \vecn\ samples are needed to characterize the prior in this space.

For the DES Y3 analyses, \citet{hyperrank} devised a scheme whereby
the samples of \vecn\ are placed in a grid within some $M$-dimensional
unit hypercube $\mathcal{H}$.  The coordinates \vecu\ within the hypercube
are considered the compressed parameters of $n(z),$ and the
decompression function
$\hat{\vecn}(\vecu)$ outputs the $\vecn_\alpha$ sample at the nearest grid point to
  \vecu.  This solves the problem of creating a continuous \vecu\
  domain, and maintains the equal prior probability of each \vecn\
  sample, but the function \emph{output,} and the resultant likelihood
  function of \vecu\, are discontinuous.
Various strategies are proposed to assign the $\vecn_\alpha$ to the grid points in
$\mathcal{H}$ based on summary statistics, to reduce the
discontinuities.  But the function is never smooth.
As a consequence, many MC samplers become quite inefficient in
sampling of the cosmological posterior.  In particular, samplers such as \textsc{MultiNest} that assume continuity are rendered nearly non-functional.  As a result, the Y3 cosmological priors could not be evaluated with this method.  Instead, the \vecn\ samples were not used, and an \textit{ad hoc} $p(\vecn)$ was adopted which allowed only shifts and dilations of the mean $n(z)$ of the \vecn\ samples.

A more rigorous and extremely efficient method of marginalizing over high-dimensional nuisance parameters was proposed by \citet{hans}, for the case where the following restrictions apply:
\begin{enumerate}
\item The likelihood of the observable \vecc\ is normal, $\vecc \sim \mathcal{N}( \hatc, \covm_c),$ with $\covm_c$ fixed.
\item The prior $p(\vecn)$ can also be assumed to be normal, with a mean and covariance matrix $\covm_n$ that in our case could be assigned from the mean and covariance of the samples of \vecn\ we are given.
\item The model \hatc\ can be linearized in \vecn\ about fiducial values $\vecq_0, \vecn_0$ without loss of accuracy exceeding measurement errors, with the derivatives independent of \vecq.
\end{enumerate}
Under these conditions, \citet{hans} show that the marginalization over \vecn\ is equivalent to adding terms to $\covm_c,$ such that any MC process need not sample \vecn\ at all.

We describe here an approach that is algebraically similar to that of \citet{hans}, but does not require 2nd condition of Gaussianity for the nuisance prior, and is somewhat more robust to non-linearities in the model that violate the 3rd condition.  Our approach is to seek a linear compression of \vecn\ into a lower-dimensional set of parameters \vecu\ that project away variations in \vecn\ that do not influence the likelihood $\likeli.$  Standard density estimators can then be applied to the \vecu\ values implied by the known \vecn\ samples to yield a prior $p(\vecu)$ that can be used for the MC chain of the cosmological posterior.  The model \hatc, and hence $\likeli,$ will be continuous over this low-dimensional \vecu\ space, and marginalization over \vecu\ will yield posterior probabilities very close to marginalization over the original \vecn.  

\section{Derivation}\label{sec:deriv}
We assume that we do have a multivariate normal likelihood for the
observables \vecc\, with the mean being some model
$\hatc(\vecq,\vecn)$ and a fixed covariance matrix $\covm_c.$ In this
section we will assume that the \vecn\ vectors have been shifted by
the mean $\vecn_0 \equiv \left\langle \vecn_\alpha \right\rangle$ so
that their mean is zero.
We seek some function $\hat{\vecn}(\vecu)$ of a lower-dimensional vector \vecu\ which can be substituted for \vecn\ and yield nearly the same likelihood function for any \vecn\ in the domain spanned by the samples $\vecn_\alpha,\; \alpha\in 1\ldots N_\alpha,$ that we are provided.  This means we want a map $\vecn_\alpha\rightarrow \vecu_\alpha$ and a function $\hat{\vecn}(\vecu)$ which minimize the error in log-likelihood
\begin{equation} -2 \left\langle \log \likeli(\vecc | \vecq,\vecn) - \log \likeli(\vecc | \vecq,\hat{\vecn}(\vecu)) \right\rangle =
  \left\langle \Delta\chi^2 \right\rangle \approx \frac{1}{N_\alpha} \sum_\alpha
                                            \left[ \hatc(\vecq,\vecn_\alpha) - \hatc(\vecq,\hat{\vecn}(\vecu_\alpha)) \right]^T
                                            \covm_c^{-1}
                                            \left[ \hatc(\vecq,\vecn_\alpha) - \hatc(\vecq,\hat{\vecn}(\vecu_\alpha)) \right].
                                          \end{equation}
                                          
If we next define the Jacobian matrix
\begin{equation}
  \jac \equiv \left.\frac{\partial\hatc}{\partial\vecn}\right|_{\vecq_0, \vecn_0}
\label{eq:jacobian}
\end{equation}
and assume that a first-order Taylor expansion about the nominal values is valid across the domain of interest, then we obtain
\begin{equation}
  \left\langle \Delta\chi^2 \right\rangle \approx \frac{1}{N_\alpha} \sum_\alpha
  \left[ \vecn_\alpha - \hat{\vecn}(u_\alpha)\right]^T \jac^T \covm_c^{-1} \jac
  \left[ \vecn_\alpha - \hat{\vecn}(u_\alpha)\right].
  \label{eq:linearized}
\end{equation}
We ask what is the optimal \emph{linear} model $\hat{\vecn} = \matU \cdot \vecu$ to use.
It is useful to define the matrix \matD, its eigensystem
$(\matV_D,\Lambda_D),$  \ and its ``square root'' $\DD$ by
\begin{equation}
   \matD \equiv \jac^T \covm_c^{-1} \jac =  \matV_D \Lambda_D
   \matV_D^T = \left(\matV_D \Lambda_D^{1/2}\right)
   \left(\Lambda_D^{1/2}\matV_D^T \right) = \DD^T \DD.
 \end{equation}
Since \matD\ is non-negative and symmetric, $\DD$ will exist and be
real-valued with non-negative singular values.
If \vecn\ has $N$ elements, and we want to compress the representation
to a vector \vecu\ with $M$ elements, we will use the $M\times N$ projection
matrix $\proj_M$ whose elements are
\begin{equation}
  \left(\proj_M\right)_{ij} \equiv
\begin{cases}
                                            1,  &  i=j\le M \\
                                            0,  & \text{otherwise.}
\end{cases}
\end{equation}
such that $\proj_M \proj_M^T=\ident_M,$ the $M\times M$ identity
matrix. We will assume that \matU\ satisfies
\begin{equation}
  \DD \matU = \matV_U \proj_M^T
\label{eq:vupm}
\end{equation}
for some unitary matrix $\matV_U$.  We can do this because, for any $\matU$, we have the
singular-value decomposition $\DD \matU = \matV_U P_M^T \Lambda_U
W_U^T,$ where $\Lambda_U$ is diagonal, $W_U$ is unitary, and both are
$M\times M$.  We can always absorb these last two factors into the \vecu\ values
to conserve the values of $\hat{\vecn} = \matU \vecu$ while making
\eqq{eq:vupm} hold.

With these definitions and a choice of \matU, the $\chi^2$ error and the values of $\vecu_\alpha$ that minimize it become
\begin{align}
  \left\langle \Delta\chi^2 \right\rangle & \approx \frac{1}{N_\alpha} \sum_\alpha
                                            \left[ \vecn_\alpha - \matU \vecu_\alpha\right]^T \matD
                                            \left[ \vecn_\alpha - \matU \vecu_\alpha\right] \\
  \Rightarrow \quad \vecu_{\alpha,\text{min}} & = \left[ \matD -
                                                \matD\matU\left(\matU^{T}\matD\matU\right)^{-1}
                                                \matU^T\matD\right]\vecn_\alpha  \\
                                          & = \proj_M \matV_U^T \DD \vecn_\alpha.
\end{align}                                            
Defining the covariance matrix of the samples $\vecn_\alpha$ as
$\covm_n=\left\langle \vecn_\alpha \vecn^T_\alpha \right\rangle,$ and
plugging the last line back into the first,  gives us the quantity we
want to minimize by selecting \matU\ properly:
\begin{align}
  \left\langle \Delta \chi^2 \right\rangle 
  & = \trace\left[\covm_n\matD
    - \covm_n \DD^T \matV_U \proj_M^T \proj_M \matV_U^T \DD \right], \\
 & = \trace\left[\matV_A \Lambda_A \matV_A^T 
    - \Lambda_A  \matV_A^T \matV_U \proj_M^T \proj_M \matV_U^T
   \matV_A\right], \\
  & = \sum_{i=1}^N \lambda^A_i \left[ 1 - \sum_{j=1}^M (\matV^T_A
    \matV_U)^2_{ij}\right], \label{eq:dchi2} \\
  \text{with} \quad \matA & \equiv \DD \covm_n \DD^T = \matV_A \Lambda_A \matV_A^T.
\end{align}
The last line defines a new symmetric, non-negative matrix \matA\ that
has eigenvectors $\matV_A$ and positive eigenvalues $\Lambda_A=\text{Diag}(\lambda^A_1,
\lambda^A_2, \ldots, \lambda^A_N).$ We assume the eigenvalues are
placed into decreasing order.

The minimization of $\left\langle \chi^2 \right\rangle$ thus requires
choosing $\matV_U,$ to minimize \eqq{eq:dchi2}, then retrieving \matU\
from \eqq{eq:vupm}.  Since $\matV_A^T\matV_U$ is a unitary matrix,
each term in brackets must be non-negative and $\le 1,$ and their total must equal
$N-M$.  The optimal case would be to have the bracketed term be zero
for $i\le M$, and be unity when multiplying the smaller $\lambda^A_i$
at $i>M.$  This occurs when $\matV_U=\matV_A,$ which is therefore our
optimal compression in terms of reproducing the likelihood of the
data.

The desired linear compression of \vecn\ therefore has the following properties:
\begin{itemize}
\item The mean error in $2\log \likeli$ is
  \begin{equation}
    \left\langle \Delta\chi^2 \right \rangle = \sum_{i=M+1}^N
    \lambda^A_i,
  \end{equation}
  and we will want to choose the dimension $M$ of the compression to
  keep this quantity, the observable effect of the parts of \vecn\ that we 
  project away,  below unity.
  Note that the $\lambda^A_i$ values are determined by a combination of
  the covariance matrix of the samples of \vecn, along with the Jacobian
  matrix and the covariance $\covm_c$ of the observables, letting us
  know the measurement significance of each mode of variation of \vecn.

  \item The \matU\ matrix containing the $M$ basis vectors for
    reconstructing \vecn\ is
    \begin{equation}
      \matU = \DD^{-1} \matV_A \proj_M^T = \matV_D \Lambda_D^{-1/2}  \matV_A \proj_M^T.
    \end{equation}
    In inverting the (non-negative, diagonal) matrix $\Lambda_D^{1/2},$ one should null the
    inverses of elements that are $\lesssim 10^{-3}$ of the largest
    element, to avoid adding large fluctuations to \matU\ that address
    noise or roundoff-error terms that contribute negligibly to $\chi^2.$ 

  \item The compressed $\vecu_\alpha$ for a chosen sample of \vecn\ is
    \begin{equation}
      \vecu_\alpha = \proj_M \matV_A^T \DD \vecn_\alpha = \proj_M
      \matV_A^T \Lambda_D^{1/2} \matV_D^T \vecn_\alpha,
    \end{equation}
    and the covariance matrix of the compressed samples is
    \begin{equation}
      \covm_u = \left\langle \vecu \vecu^T \right\rangle = \proj_M
      \matV_A^T \DD \covm_n \DD^T \matV_A \proj_M^T = \text{Diag}(\lambda^A_1,
      \lambda^A_2, \ldots \lambda^A_M).
    \end{equation}
The components of the compressed \vecu\ coefficients are therefore
uncorrelated.  Component $u_i$ has a variance
$\lambda^A_i,$ which is also equal to the reduction in $\left\langle
  \Delta \chi^2 \right\rangle$ that we will accrue if we include basis
vector $\mathbf{U}_i$ into our linear reconstruction of \vecn.
\end{itemize}
The compression thus consists of choosing the desired number of rows
of the matrix $ \matV_A^T \Lambda_D^{1/2} \matV_D^T$, and the modes
are the corresponding columns of this matrix's inverse, with each
included mode reducing the mean compression error in mean $\chi^2$ by $\lambda^A_i.$

Upon executing the compression $\{\vecn_\alpha\} \rightarrow
\{\vecu_\alpha\},$ we may find the distribution of the samples to be
consistent with a multivariate normal---which will have the diagonal covariance
matrix $\covm_u,$ and statistical independence for the elements of
\vecu.
There is,however,  no \textit{a prior} reason that
this must be the case.  We can use any density estimator over
$\{\vecu_\alpha\}$ to define a continuous $p(\vecu)$ for a Markov
chain running over the $(\vecq,\vecu)$ space.  This is the principle
advantage of our method over the single-step covariance-inflation
method of \citet{hans}.

Even if the \vecu\ prior is well approximated as multivariate normal,
there are practical advantages of compressing the nuisance 
variables and retaining them in the cosmological Markov chain rather
than using the covariance-inflation method for analytic
marginalization.  One can, for example, examine the posterior
distributions of \vecu\ and see if they are at the edges of the prior,
which would potentially indicate an inconsistency between the data and
the prior.
  

\section{Application}\label{sec:app}

xxxxxxxxx xxxxxxxxx xxxxxxxxx xxxxxxxxx xxxxxxxxx xxxxxxxxx 
xxxxxxxxx xxxxxxxxx xxxxxxxxx xxxxxxxxx xxxxxxxxx xxxxxxxxx 
xxxxxxxxx xxxxxxxxx xxxxxxxxx xxxxxxxxx xxxxxxxxx xxxxxxxxx 
xxxxxxxxx xxxxxxxxx xxxxxxxxx xxxxxxxxx xxxxxxxxx xxxxxxxxx 

  \begin{acknowledgments}


G.M.B. acknowledges support from NSF grant AST-2205808 and \ldots.

\end{acknowledgments}
\begin{thebibliography}{dummy}
%\bibliography
%\bibliographystyle{aasjournal}
\bibitem[Cordero et al.(2022)]{hyperrank} Cordero, J.~P., Harrison, I., Rollins, R.~P., et al.\ 2022, \mnras, 511, 2170. doi:10.1093/mnras/stac147

\bibitem[Hadzhiyska et al.(2020)]{hans} Hadzhiyska, B., Alonso, D., Nicola, A., et al.\ 2020, \jcap, 2020, 056. doi:10.1088/1475-7516/2020/10/056
\end{thebibliography}

\end{document}
