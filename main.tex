% section/paper on sampling with sampled priors.
% \documentclass[preprint,linenumbers]{aastex631} # 6.31 fails with \align!!
\documentclass[linenumbers, onecolumn]{aastex63}
\usepackage{xcolor}

\usepackage{natbib}
\usepackage{amsmath}
\usepackage{enumitem}

\shortauthors{Bernstein et al.}

\newcommand{\ie}{\textit{i.e.}}
\newcommand{\eg}{\textit{e.g.}}
\newcommand{\E}{\mathrm{E}}
\newcommand{\eqq}[1]{Equation~(\ref{#1})}
\newcommand\gary[1]{{\color{red} \{\textbf{GMB}: #1\}}}
%\tabletypesize{\footnotesize}
\newcommand{\vecc}{\ensuremath{\mathbf{c}}}
\newcommand{\vecq}{\ensuremath{\mathbf{q}}}
\newcommand{\vecn}{\ensuremath{\mathbf{n}}}
\newcommand{\vecu}{\ensuremath{\mathbf{u}}}
\newcommand{\vecv}{\ensuremath{\mathbf{v}}}
\newcommand{\hatc}{\ensuremath{\hat{\mathbf{c}}}}
\newcommand{\vecP}{\ensuremath{\mathbf{P}}}
\newcommand{\covm}{C}
\newcommand{\matA}{\ensuremath{A}}
\newcommand{\matD}{D}
\newcommand{\matE}{E}
\newcommand{\matF}{F}
\newcommand{\matG}{G}
\newcommand{\matI}{I}
\newcommand{\matT}{T}
\newcommand{\matX}{X}
\newcommand{\matY}{Y}
\newcommand{\matV}{V}
\newcommand{\matLam}{\Lambda}
\newcommand{\proj}{P}  % Projection matrix
\newcommand{\jac}{J}   % Jacobian
\newcommand{\ident}{I}  % identity
\newcommand{\DD}{\Delta_D}
\newcommand{\likeli}{\mathcal{L}}
\newcommand{\trace}{\text{Tr}}
\begin{document}

\title{Sampling Bayesian probabilities given only sampled priors}

\author[0000-0002-8613-8259]{Gary M. Bernstein}
\affiliation{Department of Physics and Astronomy, University of Pennsylvania, Philadelphia, PA 19104, USA}
\email{garyb@upenn.edu}

\author{Troxel,\ldots}

\begin{abstract}
	\vspace{0.2in}
Later.
\end{abstract}
\reportnum{}

\section{Motivation} \label{sec:intro}

Our motivating situation is that we have a vector of observable summary statistics \vecc, such as the binned 2-point correlation functions of cosmic fields, that we are using to constrain a set of parameters of interest \vecq, which are cosmological parameters such as $\Omega_m, \sigma_8,$ etc.  There is a model $\hatc(\vecq,\vecn)$ for the observables which involves the parameters of interest, but also a vector \vecn\ of nuisance parameters, which include the coefficients of some linear expansion of the redshift distributions $n(z)$ of some population of the galaxies being observed:
\begin{equation}
  n(z) = \sum_{k=1}^{M} n_k b_k(z).
  \label{eq:nzbasis}
\end{equation}
The $b_k$ are a set of predetermined basis functions for the redshift distribution.  In our case there are multiple galaxy populations to be characterized, leading to hundreds of parameters $n_k$ to be considered.

We wish to characterized the Bayesian posterior probability
\begin{equation}
  p(\vecq | \vecc) \propto \int dn\, \likeli(\vecc | \vecq, \vecn) p(\vecq) p(\vecn),
\label{eq:posterior}
\end{equation}
where $\likeli(\vecc | \vecq, \vecn)$ is a known likelihood function
of the data, and $p(\vecq)$ and $p(\vecn)$ are priors on the
parameters.  This posterior is complex enough that it requires
approximation by the output of a Markov chain (MC) wandering across the space $(\vecq,\vecn).$

The scenario of interest is when \emph{the prior $p(\vecn)$ is itself known only from a set of samples of $\vecn$ from this distribution.} Most MC samplers require that the posterior be an evaluable function of any value of the parameters, and it is the general task of density estimators to convert the samples of $\vecn$ into an evaluable $p(\vecn).$  But when \vecn\ is of high dimension, two problems arise: first, there may be insufficient available samples to create a viable density estimator; second, sampling of the posterior in (\ref{eq:posterior}) becomes infeasible if the MC must traverse a high-dimensional space.

One approach would be to run a new MC chain over \vecq\ for each of the samples we have of \vecn, and then concatenate these to effect marginalization over \vecn.  This is clearly infeasible if a large number of \vecn\ samples are needed to characterize the prior in this space.

For the DES Y3 analyses, \citet{hyperrank} devised a scheme whereby
the samples of \vecn\ are placed in a grid within some $M$-dimensional
unit hypercube $\mathcal{H}$.  The coordinates \vecu\ within the hypercube
are considered the compressed parameters of $n(z),$ and the
decompression function
$\hat{\vecn}(\vecu)$ outputs the $\vecn_\alpha$ sample at the nearest grid point to
  \vecu.  This solves the problem of creating a continuous \vecu\
  domain, and maintains the equal prior probability of each \vecn\
  sample, but the function \emph{output,} and the resultant likelihood
  function of \vecu, are discontinuous.
Various strategies are proposed to assign the $\vecn_\alpha$ to the grid points in
$\mathcal{H}$ based on summary statistics, to reduce the
discontinuities.  But the function is never smooth.
As a consequence, many MC samplers become quite inefficient in
sampling of the cosmological posterior.  In particular, samplers such as \textsc{MultiNest} that assume continuity are rendered nearly non-functional.  As a result, the Y3 cosmological priors could not be evaluated with this method.  Instead, the \vecn\ samples were not used, and an \textit{ad hoc} $p(\vecn)$ was adopted which allowed only shifts and dilations of the mean $n(z)$ of the \vecn\ samples.

A more rigorous and extremely efficient method of marginalizing over high-dimensional nuisance parameters was proposed by \citet{hans}, for the case where the following restrictions apply:
\begin{enumerate}
\item The likelihood of the observable \vecc\ is normal, $\vecc \sim \mathcal{N}( \hatc, \covm_c),$ with $\covm_c$ fixed.
\item The prior $p(\vecn)$ can also be assumed to be normal, with a mean and covariance matrix $\covm_n$ that in our case could be assigned from the mean and covariance of the samples of \vecn\ we are given.
\item The model \hatc\ can be linearized in \vecn\ about fiducial values $\vecq_0, \vecn_0$ without loss of accuracy exceeding measurement errors, with the derivatives independent of \vecq.
\end{enumerate}
Under these conditions, \citet{hans} show that the marginalization over \vecn\ is equivalent to adding terms to $\covm_c,$ such that any MC process need not sample \vecn\ at all.

We describe here an approach that is algebraically similar to that of \citet{hans}, but does not require 2nd condition of Gaussianity for the nuisance prior.  Our approach is to seek a linear compression of \vecn\ into a lower-dimensional set of parameters \vecu\ that project away variations in \vecn\ that do not influence the likelihood $\likeli.$  Standard density estimators can then be applied to the \vecu\ values implied by the known \vecn\ samples to yield a prior $p(\vecu)$ that can be used for the MC chain of the cosmological posterior.  The model \hatc, and hence $\likeli,$ will be continuous over this low-dimensional \vecu\ space, and marginalization over \vecu\ will yield posterior probabilities very close to marginalization over the original \vecn.  

\section{Derivation}\label{sec:deriv}
We assume that we do have a multivariate normal likelihood for the
observables \vecc\, with the mean being some model
$\hatc(\vecq,\vecn)$ and a fixed covariance matrix $\covm_c.$ In this
section we will assume that the \vecn\ vectors have been shifted by
the mean $\vecn_0 \equiv \left\langle \vecn_\alpha \right\rangle$ so
that their mean is zero.
We seek some function $\hat{\vecn}(\vecu)$ of a lower-dimensional
vector \vecu\ which can be substituted for \vecn\ and yield nearly the
same likelihood function for any \vecn\ in the domain spanned by the
samples $\vecn_\alpha,\; \alpha\in 1\ldots N_\alpha.$
This means we want a map $\vecn_\alpha\rightarrow
\vecu_\alpha \rightarrow \hat{\vecn}_\alpha$ such that replacing $\vecn$ with
$\hat\vecn$ alters the cosmological inference by much less than the
other uncertainties.  We will implement this by minimizing the
distance in the space $\vecc$ between the
model generated by $\vecn$ and that by $\hat\vecn$, using the
observations' covariance matrix $\covm_c$ as a metric for the
distance.  This is equivalent to the $\chi^2$ of the difference
between the original and compressed models for the data:
\begin{equation} \left\langle \chi^2 \right\rangle
=  \frac{1}{N_\alpha} \sum_\alpha
                                            \left[ \hatc(\vecq,\vecn_\alpha) - \hatc(\vecq,\hat{\vecn}_\alpha) \right]^T
                                            \covm_c^{-1}
                                            \left[ \hatc(\vecq,\vecn_\alpha) - \hatc(\vecq,\hat{\vecn}\alpha) \right].
\label{eq:chisq}
\end{equation}
If the data are in fact drawn from the model $\hatc(\vecq,\vecn)$ with
a Gaussian likelihood, then this is also the mean shift in $-2\log\likeli$
from the compression.  It is \emph{not}, however, equal to the mean
shift of the overall log of the posterior in
\eqq{eq:posterior}---rather, $\left\langle\chi^2\right\rangle$ is serving as a
proxy for the true log-likelihood shift. 

We next assume that the compression is linear, $\hat\vecn=\matX\vecn,$
for some matrix that is idempotent ($\matX\matX = \matX$).  If this is
true, then the contribution to \eqq{eq:chisq} at lowest order in
$\vecn-\hat\vecn$ is
\begin{equation}
  \left\langle \chi^2 \right\rangle = \frac{1}{N_\alpha} \sum_\alpha
  \left[ (\matI-\matX)\vecn_\alpha\right]^T \matF  \left[ (\matI-\matX)\vecn_\alpha\right]
  \label{eq:linearized}
\end{equation}
where we use the Jacobian matrix of the model $\hatc$ to define
\begin{equation}
  \matF \equiv
  \left[\frac{\partial\hatc}{\partial\vecn}\right]_{\vecq_0, \vecn_0}
  \covm_c^{-1} \left[\frac{\partial\hatc}{\partial\vecn}\right]_{\vecq_0,
    \vecn_0}^T.
\label{eq:fisher}
\end{equation}
This quantity is also the Fisher matrix over $\vecn$ giving the information
provided by the observations $\vecc.$  In many cases this matrix will be
rank-deficient and/or poorly conditioned, since the observables are not
likely to be very informative on $\vecn$---if they were, we might not be
concerned with establishing a prior on $\vecn$ to begin with.  We will
not, however, require the inverse of $\matF$ to proceed.

The optimization implied by \eqq{\eq:linearized} is the same as in
familiar Principal Components Analysis (PCA), aside from the presence
of the $F$ matrix, which in essence defines a new metric for the
variance to be captured by the principal components.  Our solution will follow the typical derivation
for PCA, but with an additional variable transformation to compensate
for the presence of $\matF.$

Since $\matX$ is idempotent, we can write
\begin{align}
  \matX & = V_X \proj_M V_X^T, \\
  \matY \equiv \matI-\matX & = V_X \proj_{-M} V_X^T,
\end{align}
where $V_X$ is unitary and the projection matrix $\proj_M$ is defined as
\begin{equation}
  \left(\proj_M\right)_{ij} \equiv
\begin{cases}
                                            1,  &  i=j\le M \\
                                            0,  & \text{otherwise}
\end{cases}
\end{equation}
and $\proj_{-M}=\matI-\proj_M.$  For a chosen rank $M$ of the
transformation matrix $X$, our task becomes to identify the
eigenvectors $V_X$ that minimize
\begin{align}
  \left\langle \chi^2\right\rangle & = \frac{1}{N_\alpha} \sum_\alpha
  \left[ \matY \vecn_\alpha\right]^T \matF  \left[\matY
                                     \vecn_\alpha\right] \\
       & = \trace \left[ \covm_n V_X \proj_{-M} V_X^T \matF V_X
         \proj_{-M} V_X^T \right].
\end{align}
This optimization is easier if we transform first the systematic variables to
$\vecn^\prime =\matT\vecn$ such that $\covm_{n^\prime}=\matI$, \ie\
make the elements of $\vecn$ uncorrelated and unit-variance.  This
is accomplished by finding the eigensystem $\covm_n=\matV_n \matLam_n
\matV_n^T$ and setting $\matT = \matLam_n^{-1/2} \matV_n^T$.  With
this transformation, we are now seeking a different unitary matrix
$\matV_{X^\prime}$ that minimizes
\begin{align}
  \left\langle \chi^2\right\rangle & = \trace \left[ \matI
    V_{X^\prime} \proj_{-M} V_{X^\prime}^T \left[ (T^{-1})^T \matF
      T^{-1} \right] V_{X^\prime} 
    \proj_{-M} V_{X^\prime}^T \right] \\
  & = \trace \left[ \proj_{-M} V_{X^\prime}^T \matV_G \matLam_G \matV_G^T
    V_{X^\prime} \proj_{-M} \right],
  \label{eq:mintr}\\
  \matG \equiv \left(T^{-1}\right)^T \matF   T^{-1} & = \matLam_n^{1/2} \matV_n^T
  \matF \matV_n \matLam_n^{1/2} = \matV_G \matLam_G \matV_G^T.
  \label{eq:defG}
\end{align}
The last line defines a transformed Fisher matrix $\matG$ and its
(non-negative) eigensystem.  Examination of \eqq{eq:mintr} reveals that this quantity
is minimized if the elements of the unitary matrices satisfy
$\matV_{X^\prime}^T \matV_G = \matI \; \Rightarrow \; \matV_{X^\prime} = \matV_G,$ and the eigensystem of $G$ is placed
in order of decreasing eigenvalue $\lambda^G_i.$ The
elements surviving the project functions yield
\begin{equation}
  \left\langle \chi^2\right\rangle = \sum_{i>M} \lambda_i^G.
  \label{eq:chiresid}
\end{equation}
In other words each eigenvalue of the matrix $\matG$ in \eqq{eq:defG}
gives the contribution to $\left\langle\chi^2\right\rangle$ of one
projection (mode) of $\vecn.$

Transforming the solution back into the space of $\vecn$ yields
\begin{align}
  \matX & = \matT^{-1} \matV_G \proj_M \matV_G^T \matT \\
\label{eq:DE}
   & = \left[ \matV_n^T \matLam_n^{1/2} \matV_G \proj_M \right] \left[
     \proj_M \matV_G^T \matLam_n^{-1/2} \matV_n^T \right] \\
        & \equiv \matD \matE.
\end{align}
We thus obtain our optimal encoding/compression using the nonzero rows of matrix
$\matE$ to give
\begin{equation}
  \vecu_\alpha = \matE \vecn_\alpha
  \label{eq:uu}
\end{equation}
and the decoding/reconstruction of the systematic variables as
\begin{equation}
  \hat\vecn_\alpha = \matD \vecu_\alpha.
  \label{eq:UU}
\end{equation}
One can confirm that this procedure yields a compressed representation
$\vecu$ such that $\covm_u = \matI_M,$ the $M$-dimensional
identity matrix.

The previous derivation ignores the possibility that $\covm_n$ is
singular or nearly so, such that taking $\matLam_n^{-1/2}$ in
\eqq{eq:DE} is not possible.  Indeed in our application, it is
\emph{required} that $\covm_n$ be singular, because we have a sum
normalization constraint on the initial $\vecn_\alpha$ values.  Any
such (nearly) zero element $j$ of $\matLam_n$ has a corresponding
eigenvector $\vecv_j$ of the $\vecn$ space which has zero amplitude
in all of the input samples $\vecn_\alpha,$ so that the $\vecn$ are
confined to a subspace---the reconstructed $\hat\vecn$ should also be.
The compressed
representations $\vecu_\alpha$ and reconstructed $\hat\vecn_\alpha$
should therefore be unaffected by the presence of any $\vecv_j$
component.  This can be accomplished in \eqq{eq:DE} by setting element
$j$ of $\matLam_n^{-1/2}$ to zero, as is typically done during
solutions of least-squares problems using singular value
decompositions.

In summary, the procedure for dimensional reduction is:
\begin{enumerate}
  \item Obtain the Fisher matrix $\matF$ of the system defined in
    \eqq{eq:fisher} using derivatives about the mean sample $\vecn_0,$
    plus the covariance matrix $\covm_n$ of the samples.
  \item From the eigensystem $(\matLam_n,\matV_n)$ of $C_n$, form the
    matrix $G$ defined in \eqq{eq:defG} and get its eigensystem
    $(\matLam_G,\matV_G).$  Place the eigenvalues in descending order.
  \item Choose the size $M$ of the compressed representation to be the
    minimum that keeps the $\left\langle\chi^2\right\rangle$ value in
    \eqq{eq:chiresid} below a chosen threshold, presumably $\ll 1.$
  \item The encoding matrix $E$ and decoding matrix $D$ are formed as
    in \eqq{eq:DE}, taking the inverse $\matLam_n^{-1/2}$ to be zero
    for any eigenvalues that are zero (or within roundoff errors).
  \item Compress all incoming (mean-subtracted) samples using
    \eqq{eq:uu}.  The resulting $\vecu$ values will have
    have unit covariance matrix and zero mean.
  \item Construct a continuous density estimator in $\vecu$ space that
    mimics the finite sample distribution.  If the distribution is
    normal, this becomes the multidimensional unit normal.  There
    is, however,  no \textit{a priori} reason that this must be the case,
    and something like a normalizing flow may be needed to
    approximate this lower-dimensional representation of the prior.
  \item Sample over $\vecu$ space in the Markov Chain that is sampling the
    posterior on the parameters of interest $\vecq,$ using \eqq{eq:UU}
    to transform each sample back into a $\hat\vecn$ vector.
  \end{enumerate}
  
The ability to accomodate non-Gaussian distributions of the
nuisance-parameter space is the principle
advantage of our method over the single-step covariance-inflation
method of \citet{hans}.
But even if the \vecu\ prior is well approximated as multivariate normal,
there are practical advantages of compressing the nuisance 
variables and retaining them in the cosmological Markov chain rather
than using the covariance-inflation method for analytic
marginalization.  One can, for example, examine the posterior
distributions of \vecu\ and see if they are at the edges of the prior,
which would potentially indicate an inconsistency between the data and
the prior.
  

\section{Application}\label{sec:app}
As an example of the application of this straightforward dimensional
reduction to a high-dimensional nuisance parameter, we examine the
redshift distribution of one of the bins of ``Maglim''  galaxies used
as a lens population and clustering tracer in the Year 6 analysis of
the \textit{Dark Energy Survey (DES)} galaxy catalogs.  In the baseline case,
there are 6 cosmological parameters of interest, $\vecq = \{\Omega_m,
\Omega_b, \sigma_8, h, n_s, m_\nu\}.$   The nuisance vector $\vecn$
has 5--10 parameters in each of the following categories: galaxy
biases with respect to matter; intrinsic alignments of galaxy shapes
with the tidal field of the mass; and multiplicative errors in the
measurement of galaxy shear.  The bulk of the nuisance parameters,
however, are the descriptions of the redshift distribution $n(z)$ of
galaxies belonging to each of 10 distinct color-based selections
applied to the galaxy catalog. Each galaxy sample's $n(z)$ distribution is
described by \eqq{eq:nzbasis} with 80 coefficients spanning $0<z<4$ at
intervals of $\Delta z=0.05,$  giving a total of 800 nuisance
parameters.  Each galaxy sample's $n(z)$ is also constrained to
integrate to unity, which yields 10 linear constraints on these
nuisance parameters.

Various forms of information from spectroscopic and photometric galaxy
surveys are used to constrain the $n(z)$ coefficients. As detailed in
\citet{y6pz}, these generate a set of samples of possible $n(z)$
coefficients for each galaxy sample.  We show here the results of the
mode compression process as applied to one of the nine galaxy samples,
the ??th lowest redshift of 6 lens bins.  We wish to have the
cosmological Markov chain sample from this
galaxy sample's 80-vector vector $\vecn$ of coefficients for its
redshift distribution, being guided by the available 3000??
equal-probability samples.

The observable quantities $\vecc$ whose modeling depends upon the these nuisance
parameters are the angular autocorrelation $w(\theta)$ of the sample
members, and the cross-correlations $\gamma_t^{(s)}(\theta)$ between
this sample's positions and the weak gravitational lensing shear
observed on the samples $1\le s \le 4$ of source galaxies.  A
covariance matrix spanning all of the experiment's observables is
available, and for this demonstration we will restrict it to the 30??
total elements of $\vecc$ that are affected by the $n(z)$ model for
this bin.

Figure showing violin plot, before/after.

Figure showing modes

Figure showing $\vecu$ distributions.

Describe 1d normalization mapping.

Figure showing histogram of chisq shifts.


  \begin{acknowledgments}


G.M.B. acknowledges support from NSF grant AST-2205808 and \ldots.

\end{acknowledgments}
\begin{thebibliography}{dummy}
%\bibliography
%\bibliographystyle{aasjournal}
\bibitem[Cordero et al.(2022)]{hyperrank} Cordero, J.~P., Harrison, I., Rollins, R.~P., et al.\ 2022, \mnras, 511, 2170. doi:10.1093/mnras/stac147

\bibitem[Hadzhiyska et al.(2020)]{hans} Hadzhiyska, B., Alonso, D., Nicola, A., et al.\ 2020, \jcap, 2020, 056. doi:10.1088/1475-7516/2020/10/056
\end{thebibliography}

\bibiten[TBD]{y6pz} Boyan, William, ??? et al.\ 2025 (in preparation)
Y6 redshift paper(s).

\end{document}
